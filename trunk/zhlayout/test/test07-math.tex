%#!lualatex
\documentclass{article}
\usepackage[a4paper]{geometry}
\usepackage{luatexja}
\usepackage{unicode-math}
\setmathfont{XITSMath}

% only for testing
\SetMathAlphabet{\mathgt}{bold}{JY3}{mc}{m}{n} 

\begin{document}\makeatletter
\paragraph{Unicodeの数式領域}\

標準 $a_i=i,\ i=1, 2, \dots, n$. 
\ltjdefcharrange{2}{"10000-"1FFFF}% 第1面を欧文扱いに
\ltjsetparameter{jacharrange={-2}}%
第1面を欧文扱いに $a_i=i,\ i=1, 2, \dots, n$.

\paragraph{和文数式 on \LaTeX}
数式フォント関連のコマンドは一緒.e.g.\ in \verb+lltjdefs.sty+,
\begin{verbatim}
\DeclareSymbolFont{mincho}{JY3}{mc}{m}{n}
\jfam\symmincho
\SetSymbolFont{mincho}{bold}{JY3}{gt}{m}{n}
\DeclareSymbolFontAlphabet{\mathmc}{mincho}
\DeclareMathAlphabet{\mathgt}{JY3}{gt}{m}{n}
\end{verbatim}

添字:${あz}^{いy}_{うu\mathgt{え}}$

数式内の空白処理:$a()a\hbox{a()a}a$

mathgt: $\mathrm{\mathmc{あa}}\mathmc{あb}\mathgt{あa}$
\begin{itemize}
\item 和文数式フォント選択命令は,和文文字しか影響しない
\item 欧文数式フォント選択命令は,欧文文字しか影響しない
\end{itemize}

\bf mathversion bold: {\mathversion{bold}$aあa\mathgt{あa}$}き\\
二つ目の「あ」は\verb+\mathgt+下だが,明朝なのは本文書の設定通り.

\LaTeX では数式ファミリ番号は欧文と共用→$\the\jfam$


\end{document}

